\documentclass[
  a4paper
]{article}

\usepackage{
  mlmodern,
  microtype
}

\newcommand{\acr}[1]{\textsc{\MakeLowercase{#1}}}

\pagenumbering{gobble}

\title{Quotation Management and Pertinent Information Tracking System with Expenses
Analysis for an Institution}

\author{
  Deltin~C.~K., \texttt{IDK23CS023}      \\
  Hathim~Ali~K.~H., \texttt{IDK23CS031}  \\
  Roshan~Jose, \texttt{IDK23CS049}       \\
  Srishanth~S., \texttt{IDK23CS060}
}

\date{%
  Dept. of Computer Science and Engineering \\
  Govt. Engineering College Idukki          \\[2ex]
  \itshape\today
}

\begin{document}
\maketitle

Quotation Management and Pertinent Information Tracking System with Expense Analysis for
an Institution (hereafter referred as \acr{QMS}) aims to trivialize the procedures
involved in requesting a quotation, reaching a mutual agreement, and the exchange of
considerations for an institution. When the institution posts a quotation, the relevant
registered vendors are notified who respond with inventory specifics and a prescribed
price (open to negotiation through \acr{QMS}). The institution exercises its discretion
to choose a vendor, which the vendor acknowledges and receives payment (conducted within
\acr{QMS} with a payment gateway) and prompty dispatches the item of concern. To ensure that
the item was delivered, an \acr{OTP} is generated by the vendor for the institution with
\acr{QMS} which the institution must supply to the software (assuming the sane course of
action is to supply the \acr{OTP} only once the items have indeed arrived). The complete
transaction in detail and history are archived by \acr{QMS}, these data being available
for later analyses (to draw financial conclusions and audits \&c.), and are exportable.

\vspace{2em}
\begin{flushright}
  Dr.~Sunil~K.~S.
\end{flushright}
\end{document}

\mainlanguage[en]

\setuppapersize[A4]

\setupbodyfont[11pt]

\setuplayout[backspace=4cm, cutspace=4cm, width=fit]

\setuppagenumbering[location=right, alternative=singlesided]

\setuphead[section][number=yes, sectionstopper=., conversion=romannumerals]

\setupindenting[yes, medium, next]

\setupbodyfontenvironment[default][em=italicface]

\starttext

\startalignment[middle]
{\tfb{Quotation Management and Pertinent Information Tracking System
with Expenses Analysis for a Government Institution}}
\stopalignment

\godown[2em]

\midaligned{
  \starttabulate[|lT|l{\f\sc}|lT|]
  \NC   23    \NC   Deltin~C.~K.      \NC   IDK23CS023  \NC\NR
  \NC   30    \NC   Hathim~Ali~K.~H.  \NC   IDK23CS031  \NC\NR
  \NC   48    \NC   Roshan~Jose       \NC   IDK23CS049  \NC\NR
  \NC   59    \NC   Srishanth~S.      \NC   IDK23CS060  \NC\NR
  \stoptabulate
}

\godown[2em]

\startalignment[middle]
\startlines
Dept.\ of Computer Science and Engineering
Govt.\ Engineering College Idukki

\italic{\currentdate[day,month,year]}
\stoplines
\stopalignment

\godown[2em]

Quotation Management and Pertinent Information Tracking System with Expense Analysis for a
Government Institution (hereafter referred as \cap{QMS}) aims to trivialize the procedures
involved in requesting a quotation, reaching a mutual agreement, and the exchange of
considerations for an institution. When the institution posts a quotation, the relevant
registered vendors are notified who respond with inventory specifics and a prescribed
price (open to negotiation through \cap{QMS}). The institution exercises its discretion to
choose a vendor, which the vendor acknowledges and receives payment (conducted within
\cap{QMS} with a payment gateway) and prompty dispatches the item of concern. To ensure
that the item was delivered, an \cap{OTP} is generated by the vendor for the institution
with \cap{QMS} which the institution must supply to the software (assuming the sane course
of action is to supply the \cap{OTP} only once the items have indeed arrived). The
complete transaction in detail and history are archived by \cap{QMS}, these data being
available for later analyses (to draw financial conclusions and audits \&c.), and are
exportable. The frontend is achieved using \type{Next.js/React}, the backend using
\type{Python/Django}, and \type{SQLite} or \type{PostgreSQL} as the database management
system in anticipation of scaling issues.

\blank

The primary focus of this endeavour is to electronically \emph{formalize} the sequence of
procedures comprising a usual invitation of quotation and the ensuing transactional
technicalities.  More concretely, we wish to
\startitemize[a, text, nostopper]
  \item easily notify interested (i.e.\ registered) vendors;
  \item collect quotations and painlessly review each of them because quotations here are
    not enclosed letters or like documents;
  \item archive the points of reason come up during consideration for provision under
    \cap{RTI~2005};
  \item guarantee payments and goods hand-over and record;
  \item archive the complete transaction and make it available for later review and
    analyses (for the accountants, managers, \&c.).
\stopitemize

\blank

We assume that the vendors interested in obtaining a quotation from a government
institution will register themselves in this software system under the institution’s name,
because even without it these vendors are always turning an ear to notifications released
from the institution through other media, say, the institution’s website or classifieds. A
vendor registering should ideally go through a procedure of submitting legal documents and
proofs relating to their practices, incorporation, \&c., however this is not a primary
focus and is deferred. These registered vendors are notified through the software (or
maybe also mail) when a quotation is invited by the institution, and they will send in
their quotation within this software if they wish to obtain a contract.


The review of submissions are had after a set interval wherein these vendors are to submit
their quotations. After the lapse of this interval, the submissions are considered and to
tolerate any late ones falls under the discretion of the institution. These can be classified
into four according to the requirements and preferred price of the institution:
\startitemize[indenting=yes, alignsymbol=yes]
  \item $RP$    — requirements and price met;
  \item $RP'$   — requirements met, but not price;
  \item $R'P$   — price met, but not requirements;
  \item $R'P'$  — requirements and price not met.
\stopitemize
\noindent
No further consideration is done upon $R'P'$ and they are eliminated. The other three
categories are more nuanced, for instance, the institution might prefer a submission that
deviates lightly (not negligible) but has a far lower quoted price. The submissions are
listed according price as \cap{L1}, \cap{L2}, \cap{L3}, \&c., \cap{L1} being the lowest.

In the case that \cap{L1} is not chosen, a valid reason for the same must be mandatorily
supplied. When a vendor demands the decisions behind why their quotation was rejected
under \cap{RTI~2005}, the information can be expediently dispatched.

The software also takes care to guarantee that payment and delivery have occured. The
payment is guaranteed by utilizing a payment gateway; this additionally automates and
relieves difficulties associated with payments. The delivery i.e. the hand-over of the
goods of concern are guaranteed to have actually occured using an \cap{OTP} generated by
the software at the input of the vendor which is then reentered into the software by the
institution. We assume that the institution is only willing to reenter this \cap{OTP} back
into the software in the event the goods of concern have indeed been handed over.

\placefigure[force, middle]
  {Use-Case Diagram.}
  {\externalfigure[res/use-case.pdf]}

\placefigure[middle]
	{Sequence Diagram.}
	{\externalfigure[res/sequence.pdf][scale=1500, orientation=90]}

When the quotation and its transaction are finalized, the whole detailed history is
archived in the system. The history is viewable through the software for actors with
required privileges, their privilege dictating what and how much they can view. One
significant actor is the accountant, who can draw conclusions from the archived data
through primitive analysis tools provided in the software. For further advanced review,
the data are exportable in a few appropriate formats (\cap{JSON}, \cap{XML}, \&c.); they
can be fed into other software tailored for such uses.

\blank

What this software system attempts is a digital formalization of the quotation process in
a government institution and ease the pain points encountered and morever provide
crytpgraphic security, reliability, and verifiability.

\godown[6em]

\noindent
Dated \currentdate[Y,–,M,–,D] \hfill Dr~Sunil~K.~S.

\stoptext
